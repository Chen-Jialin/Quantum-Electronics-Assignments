% !Tex program = pdflatex
% 第 9 章: 激光振荡
\ifx\allfiles\undefined
\documentclass{note}
\begin{document}
\fi
\setcounter{chapter}{8}
\chapter{激光振荡}
\begin{exe}
    推导方程式 (9.3-22).
\end{exe}
\begin{pf}
    由式 (8.3-4),
    \begin{align}
        \frac{\text{感应发射速率}/\text{模式}}{\text{自发辐射速率}/\text{模式}}=\frac{h\nu V_mW_i}{K}=n_m,
    \end{align}
    其中感应发射的跃迁速率
    \begin{align}
        W_i=\frac{\lambda^2I_{\nu}}{8\pi h\nu n^2t_{\text{自发}}}g(\nu),
    \end{align}
    辐射强度
    \begin{align}
        I_{\nu}=\frac{n_mh\nu}{V_m}\frac{c}{n},
    \end{align}
    频谱增宽
    \begin{align}
        \Delta\nu=\frac{1}{g(\nu_0)}.
    \end{align}
    故
    \begin{align}
        K=\frac{h\nu V_mW_i}{n_m}=\frac{h\nu V_m}{n_m}\frac{\lambda^2g(\nu_0)}{8\pi h\nu n^2t_{\text{自发}}}I_{\nu}=\frac{h\nu\lambda^2c}{8\pi n^3t_{\text{自发}}}g(\nu_0)=\frac{h\nu c^3}{8\pi n^3\nu^2\Delta\nu t_{\text{自发}}}.
    \end{align}
\end{pf}

\begin{exe}
    推导由于高斯光束模式的横向束缚, 对激光振荡器共振频率的影响. 在共焦腔的模式和 $z_0\gg l$ 的模式之间的共振频率有什么变化 ($l$ 为谐振腔长度)?
\end{exe}
\begin{pf}
    
\end{pf}

\begin{exe}
    若法布里-珀罗谐振腔内放入极化率为 $\chi(\omega)$ 的原子介质, 试证明模之间频率间隔等于
    \[
        \omega_m-\omega_{m-1}=\frac{\pi c}{nl\left[1+\frac{\omega}{2n^2}\frac{\partial\chi'(\omega)}{\partial\omega}\right]_{\omega=\omega_m}}.
    \]
\end{exe}
\begin{pf}
    该 F-B 腔的谐振条件为
    \begin{gather}
        k_m'l=\frac{n\omega_m}{c}\left[1+\frac{\chi'(\omega_m)}{2n^2}\right]l=m\pi,\\
        \Longrightarrow\omega_m\left[1+\frac{\chi'(\omega_m)}{2n^2}\right]=\frac{m\pi c}{nl}.
    \end{gather}
    用 $m-1$ 代换 $m$ 得
    \begin{align}
        \omega_{m-1}\left[1+\frac{\chi'(\omega_m)}{2n^2}\right]=\frac{(m-1)\pi c}{nl}.
    \end{align}
    以上两式相减得
    \begin{align}
        \omega_m-\omega_{m-1}+\frac{\omega_m\chi'(\omega_m)-\omega_{m-1}\chi'(\omega_{m-1}')}{2n^2}=(\omega_m-\omega_{m-1})\left[1+\frac{\chi'(\omega)}{2n^2}+\frac{\omega}{2n^2}\frac{\partial\chi(\omega)}{\partial\omega}\right]_{\omega=\omega_m}=\frac{\pi c}{nl},
    \end{align}
    从而模式之间频率间隔为
    \begin{align}
        \omega_m-\omega_{m-1}=\frac{\pi c}{nl\left[1+\frac{\chi'(\omega)}{2n^2}+\frac{\omega}{2n^2}\frac{\partial\chi'(\omega)}{\partial\omega}\right]_{\omega=\omega_m}}.
    \end{align}
\end{pf}

\begin{exe}
    若一个光脉冲在原子介质中传播, 脉冲的中心频率等于原子的共振频率 $\omega_0$, 考虑原子介质的色散对脉冲的群速的影响, 分两种情况讨论: (a) 放大介质, (b) 吸收介质. 若不考虑烧孔效应并假定脉冲频谱比 $\Delta\nu$ 窄, 试将群速表示为洛伦兹谱线的峰值增益的函数.
\end{exe}
\begin{sol}

\end{sol}

\begin{exe}
    证明式 (9.2-14) 等效于式 (9.1-10).
\end{exe}
\begin{pf}
    式 (9.2-14):
    \begin{align}
        (\omega_l^2-\omega^2)+i\frac{\sigma\omega}{\varepsilon}=\frac{\omega^2\varepsilon_0f}{\varepsilon}(\chi'-i\chi''),
    \end{align}
    其中 $\omega=2\pi\nu$ 为激光频率, $\omega_l=2\pi\nu_m$ 为空腔谐振频率 ($\nu_m=\frac{mc}{2nl}+\frac{c}{2\pi nl}\left(\tan^{-1}\frac{z_2}{z_0}-\tan^{-1}\frac{z_1}{z_0}-\frac{\theta_{m1}+\theta_{m2}}{2}\right)$), $\frac{\sigma}{\varepsilon}=\frac{n}{c}\frac{1}{t_c}=\alpha-\frac{1}{l}\ln(r_1r_2)=\gamma_t$ ($t_c$ 为腔寿命, $\alpha$ 为介质非共振损耗常数, $r_1$,$r_2$ 分别为两面反射镜的反射率, 这里将反射镜处的损耗也包含在 $\frac{\sigma}{\varepsilon}$ 中), $\frac{\varepsilon_0}{\varepsilon}=\frac{1}{n^2}$.
    在完全填充谐振腔且均匀翻转的情况下, $f=1$, 当 $\omega\approx\omega_l$ 时, $\omega_l^2-\omega^2\approx 2\omega(\omega_l-\omega)$, 此时
    \begin{gather}
        \nu_m-\nu+i\gamma_t=\frac{\nu}{2n^2}(\chi'-i\chi''),\\
        \Longrightarrow\nu\left[1+\frac{\chi'}{2n^2}\right]-i\nu\frac{\chi''}{2n^2}-i\frac{\gamma_t}{2}=\nu_m,
    \end{gather}
    此即等效于 (9.1-10)
    \begin{align}
        e^{-i2[k'l-\tan^{-1}(z_2/z_0)+\tan^{-1}(z_1/z_0)]}r_1r_2e^{-i(\theta_{m1}+\theta_{m2})}=e^{-i2m\pi},
    \end{align}
    其中 $k'=k\left[1+\frac{\chi'}{2n^2}\right]-ik\frac{\chi''}{2n^2}-i\frac{\alpha}{2}$, $k=\frac{2\pi\nu}{c}$.
\end{pf}

\begin{exe}
    推导式 (9.3-5).
\end{exe}
\begin{pf}
    由式 (9.3-4), 平衡态下,
    \begin{align}
        \label{9.6-1}
        \frac{\mathrm{d}N_2}{\mathrm{d}t}=&R_2-\frac{N_2}{t_2}-\left(N_2-\frac{g_2}{g_1}N_1\right)W_i(\nu)=0,\\
        \label{9.6-2}
        \frac{\mathrm{d}N_1}{\mathrm{d}t}=&R_1-\frac{N_1}{t_1}+\frac{N_2}{t_{21}}+\left(N_2-\frac{g_2}{g_1}N_1\right)W_i(\nu)=0.
    \end{align}
    上面两式相加得
    \begin{gather}
        R_2-\frac{N_2}{t_2}+R_1-\frac{N_1}{t_1}+\frac{N_2}{t_{21}}=0,\\
        \label{9.6-3}
        \Longrightarrow N_1=\left(\frac{1}{t_{21}}-\frac{1}{t_2}\right)t_1N_2+(R_1+R_2)t_1.
    \end{gather}
    将上式代入式 \eqref{9.6-1} 中得
    \begin{align}
        N_2=\frac{R_2t_2+(R_1+R_2)t_1t_2\frac{g_2}{g_1}W_i}{1+\left[t_2+(1-\delta)t_1\frac{g_2}{g_1}\right]W_i(\nu)}.
    \end{align}
    利用上式和式 \eqref{9.6-3} 得
    \begin{align}
        \Delta N\equiv N_2-\frac{g_2}{g_1}N_1=\frac{R_2t_2-(R_1+\delta R_2)t_1\frac{g_2}{g_1}}{1+\left[t_1+(1-\delta)t_1\frac{g_2}{g_1}\right]W_i(\nu)}.
    \end{align}
\end{pf}
\ifx\allfiles\undefined
\end{document}
\fi