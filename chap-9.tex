% !Tex program = pdflatex
% 第 8 章: 辐射场与原子系统的相互作用
\ifx\allfiles\undefined
\documentclass{note}
\begin{document}
\setcounter{chapter}{8}
\fi
\chapter{辐射场与原子系统的相互作用}
\begin{exe}
    推导方程式 (9.3-22).
\end{exe}
\begin{pf}

\end{pf}

\begin{exe}
    推导由于高斯光束模式的横向束缚, 对激光振荡器共振频率的影响. 在共焦腔的模式和 $z_0\gg l$ 的模式之间的共振频率有什么变化 ($l$ 为谐振腔长度)?
\end{exe}
\begin{pf}
    
\end{pf}

\begin{exe}
    若法布里-珀罗谐振腔内放入极化率为 $\chi(\omega)$ 的原子介质, 试证明模之间频率间隔等于
    \[
        \omega_m-\omega_{m-1}=\frac{\pi c}{nl\left[1+\frac{\omega}{2n^2}-\frac{\partial\chi'(\omega)}{\partial\omega}\right]_{\omega=\omega_m}}.
    \]
\end{exe}
\begin{pf}
    
\end{pf}

\begin{exe}
    若一个光脉冲在原子介质中传播, 脉冲的中心频率等于原子的共振频率 $\omega_0$, 考虑原子介质的色散对脉冲的群速的影响, 分两种情况讨论: (a) 放大介质, (b) 吸收介质. 若不考虑烧孔效应并假定脉冲频谱比 $\Delta\nu$ 窄, 试将群速表示为洛伦兹谱线的峰值增益的函数.
\end{exe}
\begin{sol}

\end{sol}

\begin{exe}
    证明式 (9.2-14) 等效于式 (9.1-10).
\end{exe}
\begin{pf}
    
\end{pf}

\begin{exe}
    推导式 (9.3-15).
\end{exe}
\begin{pf}
    
\end{pf}
\ifx\allfiles\undefined
\end{document}
\fi