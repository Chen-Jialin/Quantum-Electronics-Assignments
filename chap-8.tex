% !Tex program = pdflatex
% 第 8 章: 辐射场与原子系统的相互作用
\ifx\allfiles\undefined
\documentclass{note}
\begin{document}
\setcounter{chapter}{7}
\fi
\chapter{辐射场与原子系统的相互作用}
\begin{exe}
    证明关系式 (8.5-10)
    \[
        \frac{A}{B_{21}}=\frac{8\pi h\nu^3n^3}{c^3}
    \]
    与式 (8.3-4) 一致, 按照此式有
    \[
        \frac{\text{每个模的}W_{\text{感应}}}{\text{每个模的}W_{\text{自发}}}=n,
    \]
    式中, $n=$ 这个模式的量子数.
\end{exe}
\begin{pf}
    由式 (8.3-4), 系统由初态 $\lvert 2,n_l\rangle$ 跃迁到终态 $\lvert 1,n_{l+1}\rangle$ 的速率为
    \begin{align}
        W_{21}'=\frac{2\pi e^2\omega_l}{V\varepsilon}\abs{\langle 1,n_l+1\rvert ya_l^{\dagger}\lvert 2,n_l\rangle}^2\sin^2(k_lz)\delta(E_2-E_1-\hbar\omega_l)=\frac{2\pi e^2\omega_ly_{12}^2}{V\varepsilon}(n_l+1)\sin^2(k_lz)\delta(E_2-E_1-\hbar\omega_l),
    \end{align}
    其中 $y_{12}^2=\abs{\langle 1\rvert y\lvert 2\rangle}^2$.
    由该模式的真空涨落引发的自发跃迁速率为
    \begin{align}
        W_{21,\text{spontaneous}}'=\frac{2\pi e^2\omega_l}{V\varepsilon}\abs{\langle 1,1\rvert ya^{\dagger}\lvert 2,0\rangle}^2\sin^2(k_lz)\delta(E_2-E_1-\hbar\omega_l)=\frac{2\pi e^2\omega_ly_{12}^2}{V\varepsilon}n_l\sin^2(k_lz)\delta(E_2-E_1-\hbar\omega_l).
    \end{align}
    由该模式引发的感应 (受激) 发射的跃迁速率为
    \begin{align}
        W_{21,\text{induced}}'=W_{21}'-W_{21,\text{spontaneous}}'=\frac{2\pi e^2\omega_ly_{12}^2}{V\varepsilon}n_l\sin^2(k_lz)\delta(E_2-E_1-\hbar\omega).
    \end{align}
    因此
    \begin{align}
        \frac{\text{每个模的}W_{\text{感应}}}{\text{每个模的}W_{\text{自发}}}=n,
    \end{align}
    其中 $n=$ 这个模式的量子数.

    频率为 $\omega_l$ 的模式的真空涨落均可引发自发跃迁, 故自发发射系数为上述感应发射的跃迁速率与模式密度乘积的积分
    \begin{align}
        \notag A=&\int_0^{\infty}W_{21,\text{spontaneous}}'\rho(\nu_l)\,\mathrm{d}\nu_l=\int_0^{\infty}\frac{2\pi e^2\omega_ly_{12}^2}{V\varepsilon}\sin^2(k_lz)\delta(E_2-E_1-\hbar\omega_l)\frac{8\pi\nu_l^2n^3V}{c^3}\,\mathrm{d}\nu_l\\
        =&\frac{2\pi e^2\omega_ly_{12}^2}{V\varepsilon}\sin^2(k_lz)\frac{8\pi\nu_l^2n^3V}{hc^3}.
    \end{align}
    其中 $g_2$ 为
    频率为 $\omega_l$ 的模式中仅有激光器谐振腔限定的某一确定的传播方向和偏振的模式可引发感应跃迁, 故受激发射系数为
    \begin{align}
        B_{21}=\frac{V\int_0^{\infty}W_{12,\text{induced}}'\,\mathrm{d}\nu_l}{\rho(\nu)}=\frac{\frac{2\pi e^2\omega_ly_{12}^2}{V\varepsilon}n_l\sin^2(k_lz)\frac{V}{h}}{n_lh\nu_l}.
    \end{align}
    因此
    \begin{align}
        \frac{A}{B_{21}}=\frac{8\pi h\nu^3n^3}{c^3},
    \end{align}
    与关系式 (8.5-10) 一致.\\
    (由于自发发射和受激发射的原子初能级和末能级相同, 故此处忽略能级简并的问题.)
\end{pf}

\begin{exe}
    确定频率为 $\nu_0=3\times 10^{14}$ 赫兹跃迁的峰值吸收系数 $d(\nu_0)$, 其中 $N_2\approxeq 0$, $N_1=10^{18}$ 厘米${}^{-3}$, 高斯吸收曲线的全宽度为 $400$ 厘米 ${}^{-1}$, $t_{\text{自发}}=10^{-4}$ 秒. 定义光学密度为
    \[
        \log_{10}\frac{I_{\text{入}}}{I_{\text{出}}}
    \]
    式中 $I$ 表示强度. 对于 $1$ 厘米程长的介质, 在频率 $\nu_0$ 处的光学密度是什么? 在什么温度时, 黑体辐射的感应跃迁速率等于自发发射速率?
\end{exe}
\begin{sol}
    
\end{sol}

\begin{exe}
    若 $r$ 是电子位置坐标, 电子振荡为 $r=r_0\cos(2\pi\nu t)$, 计算此电子振荡的经典寿命 $t_{\text{经典}}=\text{能量}/(\text{辐射功率})$.
\end{exe}
\begin{sol}
    此电子振荡的能量为
    \begin{align}
        E=\frac{1}{2}m_e\dot{r}_{\max}^2=\frac{1}{2}m_e(2\pi\nu r_0)^2.
    \end{align}
    平均辐射功率为
    \begin{align}
        \bar{P}=\frac{1}{4\pi\varepsilon_0}\frac{p_0^2\omega^4}{3c^2}=\frac{1}{4\pi\varepsilon_0}\frac{(er_0)^2(2\pi\nu)^4}{3c^2}.
    \end{align}
    此电子的经典寿命为
    \begin{align}
        t_{\text{经典}}=\frac{E}{\bar{P}}=\frac{3\varepsilon_0m_ec^2}{2\pi e^2\nu^2}.
    \end{align}
\end{sol}

\begin{exe}
    \begin{itemize}
        \item[(a)] 熟习跃迁的振子强度的概念.
        \item[(b)] 在 8.2 节中所描述的跃迁的振子强度是什么?
        \item[(c)] 证明在频率为 $\nu$ 时, 跃迁 $1\leftrightarrow 2$ 的振子强度 $f_{21}$ 等于 $t_{\text{经典}}/3t_{\text{自发}}$.
    \end{itemize}
\end{exe}
\begin{sol}
    
\end{sol}

\begin{exe}
    推导式 (8.1-15).
\end{exe}
\begin{pf}
    稳态下, 式 (8.1-12) 和式 (8.1-13) 可化为
    \begin{align}
        \label{8.5-1}
        \frac{\mathrm{d}\sigma_{21}}{\mathrm{d}t}=&i(\omega-\omega_0)\sigma_{21}+\frac{i\mu E_0}{2\hbar}(\rho_{11}-\rho_{22})-\frac{\sigma_{21}}{T_2}=0,\\
        \label{8.5-2}
        \frac{\mathrm{d}}{\mathrm{d}t}(\rho_{11}-\rho_{22})=&\frac{i\mu E_0}{\hbar}(\sigma_{21}-\sigma_{21}^*)-\frac{(\rho_{11}-\rho_{22})-(\rho_{11}-\rho_{22})_0}{\tau}=0.
    \end{align}
    将式 \eqref{8.5-1} 与其复共轭相加和相减得
    \begin{align}
        (\omega-\omega_0)\im\sigma_{21}-\frac{\re\sigma_{21}}{T_2}=&0,\\
        i(\omega-\omega_0)\re\sigma_{21}+\frac{i\mu E_0}{2\hbar}(\rho_{11}-\rho_{22})-\frac{i\im\sigma_{21}}{T_2}=&0,
    \end{align}
    进而解得
    \begin{align}
        \label{8.5-3}
        \im\sigma_{21}=&\frac{\frac{\mu E_0}{2\hbar}T_2(\rho_{11}-\rho_{22})}{(\omega-\omega_0)^2T_2^2+1},\\
        \label{8.5-4}
        \re\sigma_{21}=&-\frac{\frac{\mu E_0}{2\hbar}(\omega-\omega_0)T_2^2(\rho_{11}-\rho_{22})}{(\omega-\omega_0)^2T_2^2+1}.
    \end{align}
    将式 \eqref{8.5-3} 代入式 \eqref{8.5-2} 中得
    \begin{align}
        \rho_{11}-\rho_{22}=(\rho_{11}-\rho_{22})_0\frac{1+(\omega-\omega_0)^2T_2^2}{1+(\omega-\omega_0)^2T_2^2+4\Omega^2T_2\tau},
    \end{align}
    其中 $\Omega=\frac{\mu E_0}{2\hbar}$.
    再将上式回代入式 \eqref{8.5-3} 和 \eqref{8.5-4} 中得
    \begin{align}
        \im\sigma_{21}=&\frac{\Omega T_2(\rho_{11}-\rho_{22})_0}{1+(\omega-\omega_0)^2T_2^2+4\Omega^2T_2\tau},\\
        \re\sigma_{21}=&\frac{(\omega-\omega_0)T_2^2\Omega(\rho_{11}-\rho_{22})_0}{1+(\omega-\omega_0)^2T_2^2+4\Omega^2T_2\tau}.
    \end{align}
\end{pf}

\begin{exe}
    证明在饱和可忽略的极限情况下 ($\Omega=0$), 式 (8.1-19) 中的 $\chi'(\omega)$ 和 $\chi''(\omega)$ 遵守 K-K 关系.
\end{exe}
\begin{pf}
    
\end{pf}
\ifx\allfiles\undefined
\end{document}
\fi