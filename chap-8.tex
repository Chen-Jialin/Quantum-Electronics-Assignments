% !Tex program = pdflatex
% 第 8 章: 辐射场与原子系统的相互作用
\ifx\allfiles\undefined
\documentclass{note}
\begin{document}
\setcounter{chapter}{8}
\fi
\chapter{辐射场与原子系统的相互作用}
\begin{exe}
    证明关系式 (8.5-10)
    \[
        \frac{A}{B_{21}}=\frac{8\pi h\nu^3n^3}{c^3}
    \]
    与式 (8.3-4) 一致, 按照此式有
    \[
        \frac{\text{每个模的}W_{\text{感应}}}{\text{每个模的}W_{\text{自发}}}=n,
    \]
    式中, $n=$ 这个模式的量子数.
\end{exe}
\begin{pf}
    
\end{pf}

\begin{exe}
    确定频率为 $\nu_0=3\times 10^{14}$ 赫兹跃迁的峰值吸收系数 $d(\nu_0)$, 其中 $N_2\approxeq 0$, $N_1=10^{18}$ 厘米 ${}^{-3}$, 高斯吸收曲线的全宽度为 $400$ 厘米 ${}^{-1}$, $t_{\text{自发}}=10^{-4}$ 秒. 定义光学密度为
    \[
        \log_{10}\frac{I_{\text{入}}}{I_{\text{出}}}
    \]
    式中 $I$ 表示强度. 对于 $1$ 厘米程长的介质, 在频率 $\nu_0$ 处的光学密度是什么? 在什么温度时, 黑体辐射的感应跃迁速率等于自发发射速率?
\end{exe}
\begin{sol}

\end{sol}

\begin{exe}
    若 $r$ 是电子位置坐标, 电子振荡为 $r=r_0\cos(2\pi\nu t)$, 计算次电子振荡的经典寿命 $t_{\text{经典}}=\text{能量}/(\text{辐射功率})$.
\end{exe}
\begin{sol}
    
\end{sol}

\begin{exe}
    \begin{itemize}
        \item[(a)] 熟习跃迁的振子强度的概念.
        \item[(b)] 在 8.2 节中所描述的跃迁的振子强度是什么?
        \item[(c)] 证明在频率为 $\nu$ 时, 跃迁 $1\leftrightarrow 2$ 的振子强度 $f_{21}$ 等于 $t_{\text{经典}}/3t_{\text{自发}}$.
    \end{itemize}
\end{exe}
\begin{sol}
    
\end{sol}

\begin{exe}
    推导式 (8.1-15).
\end{exe}
\begin{pf}
    
\end{pf}

\begin{exe}
    证明在饱和可忽略的极限情况下 ($\Omega=0$), 式 (8.1-19) 中的 $\chi'(\omega)$ 和 $\chi''(\omega)$ 遵守 K-K 关系.
\end{exe}
\begin{pf}
    
\end{pf}
\ifx\allfiles\undefined
\end{document}
\fi