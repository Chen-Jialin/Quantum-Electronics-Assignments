% !Tex program = pdflatex
% 第 7 章: 光学谐振腔
\ifx\allfiles\undefined
\documentclass{note}
\begin{document}
\setcounter{chapter}{7}
\fi
\chapter{光学谐振腔}
\begin{exe}
    试把从光学谐振腔 ($R\sim 0.99$) 可得到的典型 $Q$ 值与微波谐振腔的 $Q$ 进行比较.
\end{exe}
\begin{pf}
    
\end{pf}

\begin{exe}
    设计一个谐振腔, $R_1=20$ 厘米, $R_2=-32$ 厘米, $l=16$ 厘米, $\lambda=10^{-4}$ 厘米.\\
    试求
    \begin{itemize}
        \item[(a)] 最小光斑尺寸 $\omega_0$;
        \item[(b)] 最小光斑的位置;
        \item[(c)] 镜面光斑尺寸 $\omega_1$, $\omega_2$;
        \item[(d)] $\omega_0$, $\omega_1$ 和 $\omega_2$ 分别与共焦腔 ($R_1=-R_2=l$) 相应值的比.
    \end{itemize}
\end{exe}
\begin{sol}
    为保证各物理量符号的规范, 不妨重设 $R_1=-20$ cm, $R_2=32$ cm.
    \begin{itemize}
        \item[(a)] 瑞利距离的平方为
        \begin{align}
            z_0^2=\frac{l(R_2-R_1-l)(l+R_1)(l-R_2)}{(2l+R_1-R_2)^2}=92.16\,\text{cm}^2.
        \end{align}
        最小光斑尺寸为
        \begin{align}
            \omega_0=\sqrt{\frac{\lambda z_0}{n\pi}}=0.0175\,\text{cm}=175\,\mu\text{m}.
        \end{align}
        \item[(b)] 最小光斑与 1 号镜面的距离为
        \begin{align}
            \abs{z_1}=\abs{\frac{1}{2}\left[R_1\pm\sqrt{R_1^2-4z_0^2}\right]}=7.2\,\text{cm 或 }12.8\,\text{cm}.
        \end{align}
        \item[(c)] 镜面光斑尺寸分别为
        \begin{align}
            \omega_1=&\omega_0\sqrt{1+\left(\frac{z_1}{z_0}\right)^2}=0.0219\,\text{cm 或 }0.0291\,\text{cm}=219\,\mu\text{m 或 }0.291\,\mu\text{m},\\
            \omega_2=&\omega_0\sqrt{1+\left(\frac{z_2}{z_0}\right)^2}=\omega_0\sqrt{1+\left(\frac{l-z_1}{z_0}\right)^2}=0.0237\,\text{cm 或 }0.0184\,\text{cm}=237\,\mu\text{m 或 }184\,\mu\text{m}.
        \end{align}
        \item[(d)] 对对称共焦腔, 瑞利距离的平方为
        \begin{align}
            z_{0,\text{confocal}}^2=\frac{l^2}{4}=64\,\text{cm}^2,
        \end{align}
        束腰半径为
        \begin{align}
            \omega_{0,\text{confocal}}=\sqrt{\frac{\lambda z_0}{\pi n}}=0.0160\,\text{cm}=160\,\mu\text{m},
        \end{align}
        镜面光斑半径为
        \begin{align}
            \omega_{1,2,\text{confocal}}=\sqrt{2}\omega_{0,\text{confocal}}=0.0226\,\text{cm}=226\,\mu\text{m}.
        \end{align}
        题设中谐振腔 $\omega_0$、$\omega_1$ 和 $\omega_2$ 与共焦腔相应值的比分别为
        \begin{align}
            \frac{\omega_0}{\omega_{0,\text{confocal}}}=&1.1,\\
            \frac{\omega_1}{\omega_{1,\text{confocal}}}=&0.97\text{ 或 }1.3,\\
            \frac{\omega_2}{\omega_{2,\text{confocal}}}=&1.05\text{ 或 }0.83.
        \end{align}
    \end{itemize}
\end{sol}

\begin{exe}
    考虑一共焦腔, $l=16$ 厘米, $\lambda=10^{-4}$ 厘米, 反射率 $R_1=R_2=0.995$. 利用图 7.7, 选择反射镜的口径, 使第一个高阶模式 (TE$_{01}$) 的总损耗超过 $3\%$. 对于选择的口径, 基模损耗有多少? 为了抑制高阶横模的振荡, 需要多大的口径?
\end{exe}
\begin{sol}
    由图 7.7, 对模式 TE$_{01}$, $3\%$ 的损耗对应 $\frac{a^2n}{\lambda l}=0.7$, 对应反射镜的口径为 $a=0.33$ mm, 故为使模式 TE$_{01}$ 的总损耗超过 $3\%$, 反射镜的口径应 $\leq 0.33$ mm.

    对选择的口径 $a=0.33$ mm, 基模损耗约为 $0.2\%$.

    为了抑制高阶横模的振荡, 需要基模的损耗远低于高阶模, 故反射镜的口径在 $a=0.33$ mm 左右, 是一个较为合理的方案.
\end{sol}

\begin{exe}
    证明稳区图 7.4 为什么不是下列的图解表示.
    \[
        0\leq\left(1-\frac{l}{R_1}\right)\left(1-\frac{l}{R_2}\right)\leq 1
    \]
    指出图 7.1 的 8 种谐振腔在稳区图上的位置.
\end{exe}
\begin{sol}
    
\end{sol}

\begin{exe}
    按照图 7.4 (或者式 (7.2-2)), 在 $\abs{R_2}=R_1$, $R_2<0$ 时, 也就是交替排列的同样会聚和发散的两个透镜, 可以得到稳定模式. 从物理上解释为什么这会导致净聚焦.\\
    提示: 考虑光线通过两种透镜时离开轴线时的距离.
\end{exe}
\begin{sol}
    
\end{sol}

\begin{exe}
    设对称共焦腔的反射镜间距为 $l$, 曲率半径为 $R$, 利用 $ABCD$ 定理推导模式的特性 (最小光斑尺寸 $\omega_0$ 和镜面光斑尺寸 $\omega_{1,2}$).\\
    提示: 证明在反射镜面处位相波阵面的曲率半径 (也是自洽光束解) 等于反射镜的曲率半径.
\end{exe}
\begin{sol}
    
\end{sol}

\begin{exe}
    若用两个反射镜 (也就是在 $z_1$ 和 $z_2$ 处分别方两个曲率半径等于 $R(z_1)$ 和 $R(z_2)$ 的反射镜) ``代替''高斯传播光束的任意两个位相波阵面, 证明由此构成的光学谐振腔是稳定腔.
\end{exe}
\begin{pf}
    
\end{pf}

\begin{exe}
    设光学谐振腔由间距为 $l$、曲率半径为 $R$ 的两个相同的反射镜和放在中间的一薄透镜 (焦距为 $f$) 所构成, 推导模式的稳定条件.
\end{exe}
\begin{sol}
    
\end{sol}

\begin{exe}
    证明由自洽场光束参量 $q$ 的表达式 (7.2-5) 可导出光束在镜面处的曲率半径, 它分别等于反射镜的曲率半径, 即 $R(z_2)=R_2$, $R(z_1)=R_1$.
\end{exe}
\begin{pf}
    
\end{pf}

\begin{exe}
    证明光束往返一次后, 复光束参量 (它的稳态值由式 (7.2-5) 决定) 的微扰 $\Delta(1/q)$ 变为 $\delta(1/q)=e^{\mp i2\theta}\Delta(1/q)$, 其中 $\cos\theta=\frac{1}{2}(A+D)\cdot\Delta(1/q)$ 与稳定性无关, $[\delta(1/q)]=\abs{\Delta(1/q)}$ 在稳定光束中满足式 (7.2-6).
\end{exe}
\begin{pf}
    
\end{pf}
\ifx\allfiles\undefined
\end{document}
\fi