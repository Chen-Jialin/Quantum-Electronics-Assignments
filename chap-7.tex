% !Tex program = pdflatex
% 第 7 章: 光学谐振腔
\ifx\allfiles\undefined
\documentclass[twoside]{note}
\begin{document}
\fi
\setcounter{chapter}{6}
\chapter{光学谐振腔}
\begin{exe}
    试把从光学谐振腔 ($R\sim 0.99$) 可得到的典型 $Q$ 值与微波谐振腔的 $Q$ 进行比较.
\end{exe}
\begin{pf}
    设激光波长 $\lambda\sim 1\,\mu$m, 腔长 $l\sim 1$ m, $R\sim 0.99$ 的光学谐振腔的 $Q$ 值的典型值为
    \begin{align}
        Q=2\pi\frac{t_c}{T}=2\pi\frac{\frac{nl}{c[\alpha l-\ln R]}}{\frac{\lambda}{c}}=2\pi\frac{nl}{-\lambda\ln R}=6.25\times 10^8.
    \end{align}

    微波波长较激光波长大 $3\sim 6$ 个数量级, 故微波谐振腔的 $Q$ 值较光学谐振腔小 $3\sim 6$ 个数量级.
\end{pf}

\begin{exe}
    设计一个谐振腔, $R_1=20$ 厘米, $R_2=-32$ 厘米, $l=16$ 厘米, $\lambda=10^{-4}$ 厘米.\\
    试求
    \begin{itemize}
        \item[(a)] 最小光斑尺寸 $\omega_0$;
        \item[(b)] 最小光斑的位置;
        \item[(c)] 镜面光斑尺寸 $\omega_1$, $\omega_2$;
        \item[(d)] $\omega_0$, $\omega_1$ 和 $\omega_2$ 分别与共焦腔 ($R_1=-R_2=l$) 相应值的比.
    \end{itemize}
\end{exe}
\begin{sol}
    为保证各物理量符号的规范, 不妨重设 $R_1=-20$ cm, $R_2=32$ cm.
    \begin{itemize}
        \item[(a)] 瑞利距离的平方为
        \begin{align}
            z_0^2=\frac{l(R_2-R_1-l)(l+R_1)(l-R_2)}{(2l+R_1-R_2)^2}=92.16\,\text{cm}^2.
        \end{align}
        最小光斑尺寸为
        \begin{align}
            \omega_0=\sqrt{\frac{\lambda z_0}{n\pi}}=0.0175\,\text{cm}=175\,\mu\text{m}.
        \end{align}
        \item[(b)] 最小光斑与 1 号镜面的距离为
        \begin{align}
            \abs{z_1}=\abs{\frac{1}{2}\left[R_1\pm\sqrt{R_1^2-4z_0^2}\right]}=7.2\,\text{cm 或 }12.8\,\text{cm}.
        \end{align}
        \item[(c)] 镜面光斑尺寸分别为
        \begin{align}
            \omega_1=&\omega_0\sqrt{1+\left(\frac{z_1}{z_0}\right)^2}=0.0219\,\text{cm 或 }0.0291\,\text{cm}=219\,\mu\text{m 或 }0.291\,\mu\text{m},\\
            \omega_2=&\omega_0\sqrt{1+\left(\frac{z_2}{z_0}\right)^2}=\omega_0\sqrt{1+\left(\frac{l-z_1}{z_0}\right)^2}=0.0237\,\text{cm 或 }0.0184\,\text{cm}=237\,\mu\text{m 或 }184\,\mu\text{m}.
        \end{align}
        \item[(d)] 对对称共焦腔, 瑞利距离的平方为
        \begin{align}
            z_{0,\text{confocal}}^2=\frac{l^2}{4}=64\,\text{cm}^2,
        \end{align}
        束腰半径为
        \begin{align}
            \omega_{0,\text{confocal}}=\sqrt{\frac{\lambda z_0}{\pi n}}=0.0160\,\text{cm}=160\,\mu\text{m},
        \end{align}
        镜面光斑半径为
        \begin{align}
            \omega_{1,2,\text{confocal}}=\sqrt{2}\omega_{0,\text{confocal}}=0.0226\,\text{cm}=226\,\mu\text{m}.
        \end{align}
        题设中谐振腔 $\omega_0$、$\omega_1$ 和 $\omega_2$ 与共焦腔相应值的比分别为
        \begin{align}
            \frac{\omega_0}{\omega_{0,\text{confocal}}}=&1.1,\\
            \frac{\omega_1}{\omega_{1,\text{confocal}}}=&0.97\text{ 或 }1.3,\\
            \frac{\omega_2}{\omega_{2,\text{confocal}}}=&1.05\text{ 或 }0.83.
        \end{align}
    \end{itemize}
\end{sol}

\begin{exe}
    考虑一共焦腔, $l=16$ 厘米, $\lambda=10^{-4}$ 厘米, 反射率 $R_1=R_2=0.995$. 利用图 7.7, 选择反射镜的口径, 使第一个高阶模式 (TE$_{01}$) 的总损耗超过 $3\%$. 对于选择的口径, 基模损耗有多少? 为了抑制高阶横模的振荡, 需要多大的口径?
\end{exe}
\begin{sol}
    由图 7.7, 对模式 TE$_{01}$, $3\%$ 的损耗对应 $\frac{a^2n}{\lambda l}=0.7$, 对应反射镜的口径为 $a=0.33$ mm, 故为使模式 TE$_{01}$ 的总损耗超过 $3\%$, 反射镜的口径应 $\leq 0.33$ mm.

    对选择的口径 $a=0.33$ mm, 基模损耗约为 $0.2\%$.

    为了抑制高阶横模的振荡, 需要基模的损耗远低于高阶模, 故反射镜的口径在 $a=0.33$ mm 左右, 是一个较为合理的方案.
\end{sol}

\begin{exe}
    证明稳区图 7.4 为什么是下列条件的图解表示.
    \[
        0\leq\left(1-\frac{l}{R_1}\right)\left(1-\frac{l}{R_2}\right)\leq 1
    \]
    指出图 7.1 的 8 种谐振腔在稳区图上的位置.
\end{exe}
\begin{sol}
    由上述稳定条件式, 当 $\frac{l}{R_1}\geq 1$, $\frac{l}{R_2}\geq 1$ 时,
    \begin{align}
        \left(\frac{l}{R_1}-1\right)\left(\frac{l}{R_2}\right)\leq 1,
    \end{align}
    构成稳定区的右上部分,
    当 $\frac{l}{R_1}\leq 1$, $\frac{l}{R_2}\leq 1$ 时,
    \begin{align}
        \left(\frac{l}{R_1}-1\right)\left(\frac{l}{R_2}-1\right)\leq 1,
    \end{align}
    构成稳定区的左下部分,
    故图 7.4 是该稳定条件的图解表示.

    图 7.1 中
    \begin{itemize}
        \item 平面平行腔对应稳区图中的点 $(0,0)$,
        \item 第一行第二个腔对应 $0\leq\frac{l}{R_1}\leq 1$, $0\leq\frac{l}{R_2}\leq 1$ 的正方形区域,
        \item 第二行第一个腔对应 $\frac{l}{R_2}\geq\frac{l}{R_1}\geq 1$, $\left(\frac{l}{R_1}-1\right)\left(\frac{l}{R_2}-1\right)\leq 1$ 的区域,
        \item 第二行第二个腔对应 $\frac{l}{R_1}\leq 0$, $0\leq\frac{l}{R_2}\leq 1$, $\left(\frac{l}{R_1}-1\right)\left(\frac{l}{R_2}-1\right)\leq 1$ 的区域,
        \item 共焦腔对应点 $(1,1)$,
        \item 共心腔对应点 $(2,2)$,
        \item 第四行第一个高损耗腔对应右上的非稳区,
        \item 第四行第二个高损耗腔对应右上的非稳区.
    \end{itemize}
\end{sol}

\begin{exe}
    按照图 7.4 (或者式 (7.2-2)), 在 $\abs{R_2}=R_1$, $R_2<0$ 时, 也就是交替排列的同样会聚和发散的两个透镜, 可以得到稳定模式. 从物理上解释为什么这会导致净聚焦.\\
    提示: 考虑光线通过两种透镜时离开轴线时的距离.
\end{exe}
\begin{sol}
    当 $R_1\geq l$ 时,
    \begin{align}
        0\leq\left(1-\frac{l}{R_1}\right)\left(1-\frac{l}{R_2}\right)=1-\left(\frac{l}{R_1}\right)^2\leq 1,
    \end{align}
    此时可以得到稳定模式.

    该交替排列的同样会聚和发散的二元周期透镜系统的光线矩阵为
    \begin{align}
        \begin{bmatrix}
            1-\frac{2l}{R_1}&l\left(2-\frac{2l}{R_1}\right)\\
            -\left[\frac{2}{-R_1}+\frac{2}{R_1}\left(1-\frac{2l}{-R_1}\right)\right]&-\left[\frac{2l}{-R_1}-\left(1-\frac{2l}{-R_1}\right)\left(1-\frac{2l}{R_1}\right)\right]
        \end{bmatrix}
    \end{align}
    高 $h$ 的平行光轴的光线, 在透镜 1 处入射, 在透镜 2 处出射的光线为
    \begin{align}
        \notag\begin{bmatrix}
            r\\
            r'
        \end{bmatrix}=&\begin{bmatrix}
            1-\frac{2l}{R_1}&l\left(2-\frac{2l}{R_1}\right)\\
            -\left[\frac{2}{-R_1}+\frac{2}{R_1}\left(1-\frac{2l}{-R_1}\right)\right]&-\left[\frac{2l}{-R_1}-\left(1-\frac{2l}{-R_1}\right)\left(1-\frac{2l}{R_1}\right)\right]
        \end{bmatrix}\begin{bmatrix}
            h\\
            0
        \end{bmatrix}\\
        =&\begin{bmatrix}
            h\left(1-\frac{2l}{R_1}\right)\\
            -\frac{4l}{R_1^2}
        \end{bmatrix}.
    \end{align}
    故光线最终将在距透镜 2
    \begin{align}
        -\frac{h\left(1-\frac{2l}{R_1}\right)}{-\frac{4l}{R_1^2}}=-\frac{h(R_1-2l)}{4l}
    \end{align}
    处于光轴相交, 故这有可能导致净聚焦.
    % 存疑
\end{sol}

\begin{exe}
    设对称谐振腔的反射镜间距为 $l$, 曲率半径为 $R$, 利用 $ABCD$ 定律推导模式的特性 (最小光斑尺寸 $\omega_0$ 和镜面光斑尺寸 $\omega_{1,2}$).\\
    提示: 证明在反射镜面处位相波阵面的曲率半径 (也是自洽光束解) 等于反射镜的曲率半径.
\end{exe}
\begin{sol}
    自洽场条件要求腔中的光场经过往返一周后能自再现. 设反射镜面处光束的复参量为 $q$. 利用 ABCD 定律,
    \begin{align}
        q=\frac{Aq+B}{Cq+D}.
    \end{align}
    对称谐振腔可等价为焦距为 $R/2$ 和 $R/2$ 的透镜相互间隔距离 $l$ 交替排列而成的二元透镜系统, 故光线矩阵
    \begin{align}
        \begin{bmatrix}
            A&B\\
            C&D
        \end{bmatrix}=\begin{bmatrix}
            1&0\\
            -\frac{2}{R}&1
        \end{bmatrix}\begin{bmatrix}
            1&l\\
            0&1
        \end{bmatrix}\begin{bmatrix}
            1&0\\
            -\frac{2}{R}&1
        \end{bmatrix}\begin{bmatrix}
            1&l\\
            0&1
        \end{bmatrix}=\begin{bmatrix}
            1-\frac{2l}{R}&2l\left(1-\frac{l}{R}\right)\\
            \frac{4}{R}\left(\frac{l}{R}-1\right)&-\frac{2l}{R}+\left(-\frac{2l}{R}+1\right)^2
        \end{bmatrix}.
    \end{align}
    反射镜面处光斑尺寸为
    \begin{align}
        \omega_{1,2}=\left(\frac{\lambda}{\pi n}\right)^{1/2}\frac{\abs{B}^{1/2}}{\left[1-\left(\frac{D+A}{2}\right)^2\right]^{1/4}}=\left(\frac{\lambda}{\pi n}\right)^{1/2}\left(\frac{lR^2}{2R-l}\right)^{1/4},
    \end{align}
    波阵面曲率半径
    \begin{align}
        \frac{2B}{D-A}=-R,
    \end{align}
    即在反射镜面处位相波阵面的曲率半径 (即自洽光束解) 等于反射镜的曲率半径, 故对称谐振腔中的稳定高斯光束的束腰位于腔心, 从而
    \begin{align}
        R=\frac{l}{2}\left[1+\left(\frac{\pi\omega_0^2n}{\lambda\frac{l}{2}}\right)^2\right],
    \end{align}
    最小光斑尺寸
    \begin{align}
        \omega_0=\frac{\lambda l}{2\pi n}\sqrt{\frac{2R}{l}-1}.
    \end{align}
\end{sol}

\begin{exe}
    若用两个反射镜 (也就是在 $z_1$ 和 $z_2$ 处分别方两个曲率半径等于 $R(z_1)$ 和 $R(z_2)$ 的反射镜) ``代替''高斯传播光束的任意两个位相波阵面, 证明由此构成的光学谐振腔是稳定腔.
\end{exe}
\begin{pf}
    设高斯光束的束腰半径为 $\omega_0$, 则 $z_1$ 和 $z_2$ 处的波阵面曲率半径分别为
    \begin{align}
        R(z_1)=&z_1\left(1+\frac{z_0^2}{z_1^2}\right),\\
        R(z_2)=&z_2\left(1+\frac{z_0^2}{z_2^2}\right),
    \end{align}
    其中 $z_0=\frac{\pi\omega_0^2n}{\lambda}$.
    该两个反射镜构成的光学谐振腔满足稳定条件:
    \begin{align}
        0\leq\left(1-\frac{z_2-z_1}{R(z_1)}\right)\left(1-\frac{z_2-z_1}{R(z_2)}\right)=\frac{(z_0^2-z_1z_2)^2}{(z_0^2+z_1^2)(z_0^2+z_2^2)}\leq 1,
    \end{align}
    故由此构成的光学谐振腔为稳定腔.
\end{pf}

\begin{exe}
    设光学谐振腔由间距为 $l$、曲率半径为 $R$ 的两个相同的反射镜和放在中间的一薄透镜 (焦距为 $f$) 所构成, 推导模式的稳定条件.
\end{exe}
\begin{sol}
    自洽场条件要求腔中的光场经过往返一周后能再自现. 该光学谐振腔等效于由焦距为 $\frac{R}{2}$ 和 $f$ 的透镜相互间隔距离 $\frac{l}{2}$ 交替排列而成的二元透镜系统.
    % 设反射镜面处光束的复参量为 $q$. 利用 ABCD 定律,
    % \begin{align}
    %     q=\frac{Aq+B}{Cq+D}.
    % \end{align}
    % 其中
    % \begin{align}
    %     \begin{bmatrix}
    %         A&B\\
    %         C&D
    %     \end{bmatrix}=\begin{bmatrix}
    %         1&0\\
    %         -\frac{2}{R}&1
    %     \end{bmatrix}\begin{bmatrix}
    %         1&\frac{l}{2}\\
    %         0&1
    %     \end{bmatrix}\begin{bmatrix}
    %         1&0\\
    %         -\frac{1}{f}&1
    %     \end{bmatrix}\begin{bmatrix}
    %         1&\frac{l}{2}\\
    %         0&1
    %     \end{bmatrix}=\begin{bmatrix}
    %         1-\frac{l}{2f}&l-\frac{l^2}{4f}\\
    %         -\frac{2}{R}+\frac{l}{Rf}-\frac{1}{f}&-\frac{l}{R}+\left(-\frac{l}{R}+1\right)\left(-\frac{l}{2f}+1\right)
    %     \end{bmatrix}.
    % \end{align}
    与 7.2 节中的推导同理, 将式 (6.8-5) 中的 $f_1$ 替换为 $f$, $f_2$ 替换为 $R/2$, $l$ 替换为 $\frac{l}{2}$, 则得该光学谐振腔的稳定条件为
    \begin{align}
        0\leq\left(1-\frac{l}{4f}\right)\left(1-\frac{l}{2R}\right)\leq 1.
    \end{align}
\end{sol}

\begin{exe}
    证明由自洽场光束参量 $q$ 的表达式 (7.2-5) 可导出光束在镜面处的曲率半径, 它分别等于反射镜的曲率半径, 即 $R(z_2)=R_2$, $R(z_1)=R_1$.
\end{exe}
\begin{pf}
    式 (7.2-5)
    \begin{align}
        \frac{1}{q}=\frac{D-A}{2B}\pm i\frac{\sqrt{1-\left(\frac{D+A}{2}\right)^2}}{B}\equiv\frac{1}{R(z_2)}-i\frac{\lambda}{\pi\omega_2^2n},
    \end{align}
    其中
    \begin{align}
        A=&1-\frac{2l}{R_1},\\
        B=&l\left(2-\frac{2l}{R_1}\right),\\
        C=&-\left[\frac{2}{R_2}+\frac{2}{R_1}\left(1-\frac{2l}{R_1}\right)\right],\\
        D=&-\left[\frac{2l}{R_2}-\left(1-\frac{2l}{R_1}\right)\left(1-\frac{2l}{R_2}\right)\right].
    \end{align}
    故
    \begin{gather}
        \frac{1}{q}=\frac{\frac{l^2}{R_2}\left(-1+\frac{l}{R_1}\right)\pm i\sqrt{\left(1-\frac{l}{R_1}\right)\left(1-\frac{l}{R_2}\right)\left(-\frac{l^2}{R_1R_2}+\frac{l}{R_1}+\frac{l}{R_2}\right)}}{l\left(1-\frac{l}{R_1}\right)},\\
        \Longrightarrow R(z_2)=R_2.
    \end{gather}
    同理可证 $R(z_1)=R_1$, 故自洽场光束在镜面处的曲率半径等于反射镜的曲率半径.
\end{pf}

\begin{exe}
    证明光束往返一次后, 复光束参量 (它的稳态值由式 (7.2-5) 决定) 的微扰 $\Delta(1/q)$ 变为 $\delta(1/q)=e^{\mp i2\theta}\Delta(1/q)$, 其中 $\cos\theta=\frac{1}{2}(A+D)\cdot\Delta(1/q)$ 与稳定性无关, $\abs{\delta(1/q)}=\abs{\Delta(1/q)}$ 在稳定光束中满足式 (7.2-6).
\end{exe}
\begin{pf}
    由式 (7.2-5), 稳态下光束的复参量
    \begin{align}
        \frac{1}{q}=\frac{D-A}{2B}\pm i\frac{\sqrt{1-\left(\frac{D+A}{2}\right)^2}}{B}.
    \end{align}
    利用 ABCD 定律, 参量受微扰 $\Delta(1/q)$ 的光束往返一次后参量变为
    \begin{align}
        \notag\frac{1}{q}+\delta\left(\frac{1}{q}\right)=&\frac{D\left[\frac{1}{q}+\Delta\left(\frac{1}{q}\right)\right]+C}{B\left[\frac{1}{q}+\Delta\left(\frac{1}{q}\right)\right]+A}=\frac{D\left[\frac{1}{q}+\Delta\left(\frac{1}{q}\right)\right]+C}{B\frac{1}{q}+A}\left[1-\frac{B\Delta\left(\frac{1}{q}\right)}{B\frac{1}{q}+A}\right]\\
        =&\frac{D\frac{1}{q}+C}{B\frac{1}{q}+A}+\frac{D\Delta\left(\frac{1}{q}\right)}{B\frac{1}{q}+A}-\frac{D\frac{1}{q}+C}{B\frac{1}{q}+A}\frac{B\Delta\left(\frac{1}{q}\right)}{B\frac{1}{q}+A}=\frac{1}{q}+\frac{D\Delta\left(\frac{1}{q}\right)}{B\frac{1}{q}+A}-\frac{1}{q}\frac{B\Delta\left(\frac{1}{q}\right)}{B\frac{1}{q}+A},
    \end{align}
    从而参量的微扰变为
    \begin{align}
        \delta\left(\frac{1}{q}\right)=\frac{D-B\frac{1}{q}}{B\frac{1}{q}+A}\Delta\frac{1}{q}=\frac{\frac{D+A}{2}\mp i\sqrt{1-\left(\frac{D+A}{2}\right)^2}}{\frac{D+A}{2}\pm i\sqrt{1-\left(\frac{D+A}{2}\right)^2}}\Delta\left(\frac{1}{q}\right)=\frac{\frac{D+A}{2}\mp i\sqrt{1-\left(\frac{D+A}{2}\right)^2}}{\left(\frac{D+A}{2}\right)^2+\left[1-\left(\frac{D+A}{2}\right)^2\right]}=e^{\mp i2\theta}\Delta\left(\frac{1}{q}\right),
    \end{align}
    其中
    \begin{align}
        \tan 2\theta=\frac{\sqrt{1-\left(\frac{D+A}{2}\right)^2}}{\frac{D+A}{2}},
    \end{align}
    即
    \begin{align}
        \cos 2\theta=\frac{D+A}{2}.
    \end{align}
    当 $\abs{\frac{D+A}{2}}\leq 1$, 即满足式 (7.2-6) 时, $2\theta$ 为实数, 从而
    \begin{align}
        \abs{\delta\left(\frac{1}{q}\right)}=\abs{\Delta\left(\frac{1}{q}\right)}.
    \end{align}
\end{pf}
\ifx\allfiles\undefined
\end{document}
\fi