% !Tex program = pdflatex
% 第 15 章: 辐射场与原子系统的相干相互作用
\ifx\allfiles\undefined
\documentclass[twoside]{note}
\begin{document}
\fi
\setcounter{chapter}{14}
\chapter{辐射场与原子系统的相干相互作用}
\begin{exe}
    证明若原子初始处于低能态 $\lvert b\rangle$, 即 $\bm{r}_3(0)=-1$, 则描述原子初始处于高能态 $\lvert a\rangle$ 的运动的 $\bm{r}(t)$ 为负值.
\end{exe}
\begin{pf}
    对于初始处于低能态 $\lvert b\rangle$ 的原子, 初始态矢量为
    \begin{align}
        \bm{r}=\begin{bmatrix}
            0\\
            0\\
            -1
        \end{bmatrix},
    \end{align}
    且态矢量运动方程为
    \begin{align}
        \frac{\mathrm{d}\bm{r}}{\mathrm{d}t}=\bm{\omega}(t)\times\bm{r},
    \end{align}
    从而解得 $\bm{r}$ 绕 $\bm{\omega}$ 逆时针转动 (逆着 $\bm{\omega}$ 的方向看).

    对于初始处于高能态 $\lvert a\rangle$ 的原子, 初始态矢量为
    \begin{align}
        \bm{r}'(0)=\begin{bmatrix}
            0\\
            0\\
            1
        \end{bmatrix}=-\bm{r}(0),
    \end{align}
    同理解得 $\bm{r}'$ 同样绕 $\bm{\omega}$ 逆时针转动.
    故从两种初始状态开始演化的态矢量始终相反, $\bm{r}(t)=-\bm{r}'(t)$.
\end{pf}

\begin{exe}
    推导式 (15.1-29) 和 (15.1-30).
\end{exe}
\begin{pf}
    矢量 $\bm{\omega}$ 与 III 轴的夹角 $\theta$ 满足
    \begin{align}
        \sin\theta=&\frac{\frac{2\mu E}{\hbar}}{\omega_e},\\
        \cos\theta=&\frac{\omega-\omega_0}{\omega_e},\\
        \tan\theta=&\frac{\frac{2\mu E}{\hbar}}{\omega-\omega_0},
    \end{align}
    其中
    \begin{align}
        \omega_e=\sqrt{\left(\frac{2\mu E}{\hbar}\right)^2+(\omega_0-\omega)^2}.
    \end{align}
    由基本的三角关系得
    \begin{align}
        r_I=&-\cos\theta\sin\theta+\sin\theta\cos\theta\cos\omega_et=-\frac{\frac{2\mu E}{\hbar}(\omega-\omega_0)}{\omega_e}(1-\cos\omega_et)=\frac{\omega_I(\omega-\omega_0)}{\omega_e^2}(1-\cos\omega_et),\\
        r_{II}=&\sin\theta\sin\omega_et=\frac{\frac{2\mu E}{\hbar}}{\omega_e}\sin\omega_et=-\frac{\omega_I}{\omega_e}\sin\omega_et,\\
        r_{III}=&\cos^2\theta+\sin^2\theta\cos\omega_et=1-\sin^2\theta(1-\cos\omega_et)=1-2\left(\frac{\frac{2\mu E}{\hbar}}{\omega_e}\right)^2\sin^2\left(\frac{\omega_et}{2}\right)=1-2\left(\frac{\omega_I}{\omega_e}\right)^2\sin^2\left(\frac{\omega_et}{2}\right),
    \end{align}
    此即课本式 (15.1-29), 其中 $\omega_I=-\frac{2\mu E}{\hbar}$.

    利用关系式 $r_{III}=\abs{a}^2-\abs{b}^2$ 和归一化条件 $\abs{a}^2+\abs{b}^2=1$ 得
    \begin{align}
        \abs{a}^2=&\frac{1+r_{III}}{2}=1-\left(\frac{\omega_I}{\omega_e}\right)^2\sin^2\left(\frac{\omega_et}{2}\right),\\
        \abs{b}^2=&\frac{1-r_{III}}{2}=\left(\frac{\omega_I}{\omega_e}\right)^2\sin^2\left(\frac{\omega_et}{2}\right).
    \end{align}
\end{pf}

\begin{exe}
    试求原子系综的感应偶极矩. 原子在场 $E_x=E_0\cos(\omega_0t-\bm{k}_0\cdot\bm{r})+E_1\cos[(\omega_0+\Delta)t-\bm{k}_1\cdot\bm{r}]$ 作用下, 其中 $E_1\ll E_0$, 原子有共振跃迁 $E_a-E_b=\hbar\omega_0$, 并初始处于基态 $\lvert b\rangle$. 假设样品的尺寸比 $\lambda_0$ 大, 证明原子沿 $2\bm{k}_0-\bm{k}_1$ 方向辐射频率为 $\omega_0-\Delta$ 的波.
\end{exe}
\begin{sol}
    原子系综的态矢量遵循运动方程
    \begin{align}
        \frac{\mathrm{d}\bm{r}}{\mathrm{d}t}=\bm{\omega}\times\bm{r},
    \end{align}
    其中
    \begin{align}
        \notag\omega_1=&-\frac{2\mu E_x(t)}{\hbar}=-\frac{2\mu}{\hbar}\{E_0\cos(\omega_0t-\bm{k}_0\cdot\bm{r})+E_1\cos[(\omega_0+\Delta)t-\bm{k}_1\cdot\bm{r}]\}\\
        =&-\frac{\mu}{\hbar}\{E_0\cos(\omega_0t-\bm{k}_0\cdot\bm{r})+E_0\cos(\omega_0t-\bm{k}_0\cdot\bm{r})+E_1\cos[(\omega_0+\Delta)t-\bm{k}_1\cdot\bm{r}]+E_1\cos[(\omega_0+\Delta)t-\bm{k}_1\cdot\bm{r}]\},\\
        \notag\omega_2=&-\frac{2\mu E_y(t)}{\hbar}=0\\
        =&-\frac{\mu}{\hbar}\{E_0\sin(\omega_0t-\bm{k}_0\cdot\bm{r})-E_0\sin(\omega_0t-\bm{k}_0\cdot\bm{r})+E_1\sin[(\omega_0+\Delta)t-\bm{k}\cdot\bm{r}]-E_1\sin[(\omega_0+\Delta)t-\bm{k}_1\cdot\bm{r}]\},\\
        \omega_3=&\omega_0,
    \end{align}
    即将 $\bm{\omega}$ 中的两个线偏振矢量各分别化为两个圆偏振矢量的叠加.
    变换到绕 3 轴以角速度 $\omega_0$ 旋转的坐标系中, 则该旋转坐标系中的态矢量 $\bm{r}_R$ 遵循运动方程
    \begin{align}
        \frac{\mathrm{d}\bm{r}_R}{\mathrm{d}t}=(\bm{\omega}_R-\bm{\omega}_0)\times\bm{r}_R,
    \end{align}
    其中
    \begin{align}
        \bm{\omega}_0=\begin{bmatrix}
            0\\
            0\\
            \omega_0
        \end{bmatrix}
    \end{align}
    在旋波近似下, $\bm{\omega}$ 旋转坐标系中的坐标为
    \begin{align}
        \bm{\omega}_R\approx\begin{bmatrix}
            -\frac{\mu}{\hbar}[E_0\cos(-\bm{k}_0\cdot\bm{r})+E_1\cos(\Delta t-\bm{k}_1\cdot\bm{r})]\\
            -\frac{\mu}{\hbar}[E_0\sin(-\bm{k}_0\cdot\bm{r})+E_1\sin(\Delta t-\bm{k}_1\cdot\bm{r})]\\
            \omega_0
        \end{bmatrix},
    \end{align}
    故运动方程变为
    \begin{align}
        \frac{\mathrm{d}\bm{r}_R}{\mathrm{d}t}=-\frac{\mu}{\hbar}\begin{bmatrix}
            E_0\cos(-\bm{k}_0\cdot\bm{r})+E_1\cos(\Delta t-\bm{k}_1\cdot\bm{r})\\
            E_0\sin(-\bm{k}_0\cdot\bm{r})+E_1\sin(\Delta t-\bm{k}_1\cdot\bm{r})\\
            0
        \end{bmatrix}\times\bm{r}_R.
    \end{align}
    $\cdots$
\end{sol}

\begin{exe}
    用任意矢量 $\bm{A}(A_1,A_2,A_3)$ 代替 $\bm{r}(t)$, 试证式 (15.1-21) 成立.\\
    提示: 取
    \[
        \bm{A}_{R}(t)=\begin{bmatrix}
            \cos\Omega t&\sin\Omega t&0\\
            -\sin\Omega t&\cos\Omega t&0\\
            0&0&1
        \end{bmatrix}\begin{bmatrix}
            A_1\\
            A_2\\
            A_3
        \end{bmatrix},
    \]
    因此
    \[
        \frac{\mathrm{d}\bm{A}_R(t)}{\mathrm{d}t}=\overline{\overline{T}}\frac{\mathrm{d}\bm{A}}{\mathrm{d}t}+\frac{\mathrm{d}\overline{\overline{T}}}{\mathrm{d}t}\bm{A},
    \]
    式中 $\overline{\overline{T}}$ 是上式中的变换矩阵.
\end{exe}
\begin{pf}
    旋转坐标系中矢量可表为
    \begin{align}
        \bm{A}_R(t)=\begin{bmatrix}
            A_1\cos\Omega t+A_2\sin\Omega t\\
            -A_2\sin\Omega t+A_2\cos\Omega t\\
            A_3
        \end{bmatrix}.
    \end{align}
    旋转坐标系中矢量的变化量可表为
    \begin{align}
        \frac{\mathrm{d}\bm{r}_R}{\mathrm{d}t}=\begin{bmatrix}
            \dot{A}_1\cos\Omega t-A_1\sin\Omega t+\dot{A}_2\sin\Omega t+A_2\cos\Omega t\\
            -\dot{A}_1\sin\Omega t-A_1\cos\Omega t+\dot{A}_2\cos\Omega t-A_2\sin\Omega t\\
            \dot{A}_3
        \end{bmatrix}.
    \end{align}
    而又因
    \begin{align}
        \notag\left(\frac{\mathrm{d}\bm{A}}{\mathrm{d}t}\right)_R-\bm{\Omega}\times\bm{A}_R=&\begin{bmatrix}
            \dot{A}_1\cos\Omega t+\dot{A}_2\sin\Omega t\\
            -\dot{A}_2\sin\Omega t+\dot{A}_2\cos\Omega t\\
            \dot{A}_3
        \end{bmatrix}+\begin{vmatrix}
            \bm{a}_I&\bm{a}_{II}&\bm{a}_{III}\\
            0&0&\Omega\\
            A_1\cos\Omega t+A_2\sin\Omega t&-A_1\sin\Omega+A_2\cos\Omega t&A_3
        \end{vmatrix}\\
        =&\begin{bmatrix}
            \dot{A}_1\cos\Omega t+\dot{A}_2\sin\Omega t\\
            -\dot{A}_2\sin\Omega t+\dot{A}_2\cos\Omega t\\
            \dot{A}_3
        \end{bmatrix}-\begin{bmatrix}
            A_1\Omega\sin\Omega-A_2\Omega\cos\Omega t\\
            A_1\Omega\cos\Omega t+A_2\Omega\sin\Omega t\\
            0
        \end{bmatrix},
    \end{align}
    故
    \begin{align}
        \frac{\mathrm{d}\bm{A}}{\mathrm{d}t}=\left(\frac{\mathrm{d}\bm{A}}{\mathrm{d}t}\right)_R-\bm{\Omega}\times\bm{A}_R,
    \end{align}
    此即课本式 (15.1-21).
\end{pf}
\ifx\allfiles\undefined
\end{document}
\fi