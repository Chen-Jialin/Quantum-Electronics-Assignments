% !Tex program = pdflatex
% 第 15 章: 辐射场与原子系统的相干相互作用
\ifx\allfiles\undefined
\documentclass{note}
\begin{document}
\fi
\setcounter{chapter}{14}
\chapter{辐射场与原子系统的相干相互作用}
\begin{exe}
    证明若原子初始处于低能态 $\lvert b\rangle$, 即 $\bm{r}_3(0)=-1$, 则描述原子初始处于高能态 $\lvert a\rangle$ 的运动的 $\bm{r}(t)$ 为负值.
\end{exe}
\begin{pf}

\end{pf}

\begin{exe}
    推导式 (15.1-29) 和 (15.1-30).
\end{exe}
\begin{pf}
    
\end{pf}

\begin{exe}
    试求原子系综的感应偶极矩. 原子在场 $E_a=E_0(\omega_0t-\bm{k}_0\cdot\bm{r})+E_1\cos[(\omega_0+\Delta)t-\bm{k}_1\cdot\bm{r}]$ 作用下, 其中 $E_1\ll E_0$, 原子有共振跃迁 $E_a-E_b=\hbar\omega_0$, 并初始处于基态 $\lvert b\rangle$. 假设样品的尺寸比 $\lambda_0$ 大, 证明原子沿 $2\bm{k}_0-\bm{k}_1$ 方向辐射频率为 $\omega_0-\Delta$ 的波.
\end{exe}
\begin{sol}

\end{sol}

\begin{exe}
    用任意矢量 $\bm{A}(A_1,A_2,A_3)$ 代替 $\bm{r}(t)$, 试证式 (15.1-21) 成立.\\
    提示: 取
    \[
        \bm{A}_{R}(t)=\begin{vmatrix}
            \cos\Omega t&\sin\Omega t&0\\
            -\sin\Omega t&\cos\Omega t&0\\
            0&0&1
        \end{vmatrix}\begin{vmatrix}
            A_1\\
            A_2\\
            A_3
        \end{vmatrix},
    \]
    因此
    \[
        \frac{\mathrm{d}\bm{A}_R(t)}{\mathrm{d}t}=\overline{\overline{A}}\frac{\mathrm{d}\bm{A}}{\mathrm{d}t}+\frac{\mathrm{d}\overline{\overline{T}}}{\mathrm{d}t}\bm{A},
    \]
    式中 $\overline{\overline{T}}$ 是上式中的变换矩阵.
\end{exe}
\begin{pf}
    
\end{pf}
\ifx\allfiles\undefined
\end{document}
\fi