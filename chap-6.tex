% !Tex program = pdflatex
% 第 6 章: 光束在均匀介质和类透镜介质中的传播
\ifx\allfiles\undefined
\documentclass{note}
\begin{document}
\setcounter{chapter}{5}
\fi
\chapter{光束在均匀介质和类透镜介质中的传播}
\begin{exe}
    推导 (6.2-1) 至 (6.2-4) 各式.
\end{exe}
\begin{pf}
    对相同透镜构成的波导, (在 $n+1$ 处) 出射和 (在 $n$ 处) 出射光线之间的关系为
    \begin{align}
        \begin{bmatrix}
            r_{n+1}\\
            r_{n+1}'
        \end{bmatrix}=\begin{bmatrix}
            1&d\\
            -1/f&(1-d/f)
        \end{bmatrix}\begin{bmatrix}
            r_n\\
            r_n'
        \end{bmatrix},
    \end{align}
    或将其写成方程形式为
    \begin{align}
        \label{6.1-1}
        r_{n+1}=&r_n+dr_n',\\
        \label{6.1-2}
        r_{n+1}'=&-\frac{1}{f}r_n+\left(1-\frac{d}{f}\right)r_n'.
    \end{align}
    由式 \eqref{6.1-1} 得
    \begin{align}
        r_n'=\frac{1}{d}(r_{n+1}-r_n),
    \end{align}
    因而有
    \begin{align}
        r_{n+1}'=\frac{1}{d}(r_{n+2}-r_{n+1}),
    \end{align}
    将以上两式代入式 \eqref{6.1-2} 中得
    \begin{align}
        \label{6.1-3}
        r_{n+2}-\left(2-\frac{d}{f}\right)r_{n+1}+r_n=0.
    \end{align}
    上式为微分方程
    \begin{align}
        r''+\frac{d}{f}r=0
    \end{align}
    的差分方程, 故令试探解为
    \begin{align}
        r_n=r_0e^{in\theta},
    \end{align}
    将其代入式 \eqref{6.1-3} 中可得
    \begin{align}
        e^{2i\theta}-\left(2-\frac{d}{f}\right)e^{i\theta}+1=0,
    \end{align}
    解得
    \begin{align}
        e^{i\theta}=\left(1-\frac{d}{f}\right)\pm i\sqrt{\frac{d}{f}-\frac{d^2}{4f^2}},
    \end{align}
    于是得到
    \begin{align}
        r_n=r_{\text{最大}}\sin\theta(n\theta+\delta),
    \end{align}
    其中
    \begin{align}
        \cos\theta=\re\left[e^{i\theta}\right]=\left(1-\frac{d}{2f}\right),
    \end{align}
    此即式 (6.2-2),
    \begin{align}
        r_{\text{最大}}=\frac{r_0}{\sin\delta}.
    \end{align}
    利用 \ref{6.1-1} 得
    \begin{gather}
        r_{\text{最大}}\sin(\theta+\delta)=r_0+dr_0',\\
        \Longrightarrow \frac{r_0}{\sin\delta}(\sin\theta\cos\delta+\cos\theta\sin\delta)=r_0+dr_0',\\
        \Longrightarrow\tan\delta=\frac{\sin\delta}{\cos\delta}=\frac{r_0\sin\theta}{(r_0+dr_0')-r_0\cos\theta}=\frac{\sqrt{\frac{4f}{d}-1}}{1+2f\frac{r_0'}{r_0}},
    \end{gather}
    此即式 (6.2-4), 从而
    \begin{align}
        (r_{\text{最大}})^2=\frac{r_0^2}{\sin^2\delta}=\frac{r_0^2(1+\tan^2\delta)}{\tan^2\delta}=\frac{4f}{4f-d}(r_0^2+dr_0r_0'+dfr_0'^2),
    \end{align}
    此即式 (6.2-3).
    光线稳定的条件为
    \begin{gather}
        \frac{d}{f}-\frac{d^2}{4f^2}\geq 0,\\
        \Longrightarrow 0\leq d\leq 4f,
    \end{gather}
    此即式 (6.2-1).
\end{pf}

\begin{exe}
    证明方程
    \[
        \begin{bmatrix}
            A&B\\
            C&D
        \end{bmatrix}\begin{bmatrix}
            r_s\\
            r_s'
        \end{bmatrix}=\lambda\begin{bmatrix}
            r_s\\
            r_s'
        \end{bmatrix}
    \]
    的本征值是 $\lambda=e^{\pm i\theta}$, 其中 $\exp(\pm i\theta)$ 由式 (6.1-13) 给出. 注意, 按照式 (6.1-5), 上述矩阵方程也可以写为
    \begin{align}
        \begin{bmatrix}
            r_{s+1}\\
            r_{s+1}'
        \end{bmatrix}=\lambda\begin{bmatrix}
            r_s\\
            r_s'
        \end{bmatrix}.
    \end{align}
\end{exe}
\begin{pf}
    上述矩阵的特征方程为
    \begin{align}
        \det\left(\begin{vmatrix}
            A&B\\
            C&D
        \end{vmatrix}-\lambda I\right)=\begin{vmatrix}
            A-\lambda&B\\
            C&D-\lambda
        \end{vmatrix}=\lambda^2-(A+D)\lambda+(AD-BC)=\lambda^2-2\lambda+1=0,
    \end{align}
    从而解得特征值为
    \begin{align}
        \lambda=e^{\pm i\theta},
    \end{align}
    其中 $\exp(\pm i\theta)$ 由式 (6.1-13) 给出:
    \begin{align}
        e^{\pm i\theta}=b\pm\sqrt{1-b^2}.
    \end{align}
\end{pf}

\begin{exe}
    对于一个平面波入射到透镜上的情况证明式 (6.4-1) 成立, 由此合理地论证式 (6.4-1) 是正确的.
\end{exe}
\begin{pf}
    设透镜的折射率为 $n$, $(x,y)$ 处厚度为 $t(\sqrt{x^2+y^2})$. 当平面波入射到透镜上, 沿光轴入射透镜和平行光轴于 $(x,y)$ 入射透镜的光线汇聚于焦点, 根据费马原理, 两者自入射到汇聚走过的光程相等:
    \begin{align}
        nt(0)+f=nt(\sqrt{x^2+y^2})+(t(\sqrt{x^2+y^2})-t(0))+\sqrt{f^2+x^2+y^2}.
    \end{align}
    傍轴近似下, 有
    \begin{gather}
        nt(0)+f\approx nt(\sqrt{x^2+y^2})+[t(\sqrt{x^2+y^2})-t(0)]+f\left(1+\frac{x^2+y^2}{2f^2}\right),\\
        \Longrightarrow nt(\sqrt{x^2+y^2})+[t(\sqrt{x^2+y^2})-t(0)]=nt(0)-\frac{x^2+y^2}{2f}.
    \end{gather}
    透镜对平面波的作用是引起相移
    \begin{align}
        \phi(\sqrt{x^2+y^2})=k\{nt(\sqrt{x^2+y^2})+[t(\sqrt{x^2+y^2})-t(0)]\}=knt(0)-k\frac{x^2+y^2}{2f},
    \end{align}
    而不影响振幅, 因此 (忽略整体相因子) 有
    \begin{align}
        E_R(x,y)=E_L(x,y)\exp\left(-ik\frac{x^2+y^2}{2f}\right),
    \end{align}
    此即式 (6.4-1).
\end{pf}

\begin{exe}
    推导方程式 (6.4-7).
\end{exe}
\begin{pf}
    $l$ 处的光线可表为
    \begin{align}
        r(l^-)=&r_0\cos\left(\sqrt{\frac{k_2}{k}}l\right),\\
        r'(l^-)=&-\sqrt{\frac{k_2}{k}}r_0\sin\left(\sqrt{\frac{k_2}{k}}l\right).
    \end{align}
    边界处出射可表为
    \begin{align}
        r(l^+)=&r(l^-)=r_0\cos\left(\sqrt{\frac{k_2}{k}}l\right),\\
        r'(l^+)=&n_0r'(l^-)=-n_0\sqrt{\frac{k_2}{k}}r_0\sin\left(\sqrt{\frac{k_2}{k}}l\right).
    \end{align}
    公共焦点与出射面之间的距离为
    \begin{gather}
        h=\abs{\frac{r_{\text{out}}}{r'_{\text{out}}}}=\frac{1}{n_0}\sqrt{\frac{k_2}{k}}\cot\left(\sqrt{\frac{k_2}{k}}l\right).
    \end{gather}
\end{pf}

\begin{exe}
    证明占据 $0\leq z\leq l$ 区域的类透镜介质把位于 $z<0$ 的轴上点成像为单点. (若成像点位于 $z<l$ 处, 则是虚像.)
\end{exe}
\begin{pf}
    设轴上物点坐标为 $-z_0<0$, 则由该点发出的某条光线可表为 $(0,r_0')$, 经 $-z_0<z<0$ 区域的直线传播后在入射面的入射光线可表为
    \begin{align}
        \begin{bmatrix}
            r(0^-)\\
            r'(0^-)
        \end{bmatrix}=\begin{bmatrix}
            1&z_0\\
            0&1
        \end{bmatrix}\begin{bmatrix}
            0\\
            r_0'
        \end{bmatrix}=\begin{bmatrix}
            z_0r'\\
            r_0'
        \end{bmatrix}.
    \end{align}
    经入射面折射后, 光线可表为
    \begin{align}
        \begin{bmatrix}
            r(0^+)\\
            r'(0^+)
        \end{bmatrix}=\begin{bmatrix}
            1&0\\
            0&\frac{1}{n_0}
        \end{bmatrix}\begin{bmatrix}
            r(0^-)\\
            r'(0^-)
        \end{bmatrix}=\begin{bmatrix}
            z_0r'\\
            \frac{r_0'}{n_0}
        \end{bmatrix}.
    \end{align}
    经类透镜介质中传播后, 在出射面处出射前的光线可表为
    \begin{align}
        \begin{bmatrix}
            r(l^-)\\
            r'(l^-)
        \end{bmatrix}=\begin{bmatrix}
            \cos\left(\sqrt{\frac{k_2}{k}}l\right)&\sqrt{\frac{k}{k_2}}\sin\left(\sqrt{\frac{k_2}{k}}l\right)\\
            -\sqrt{\frac{k_2}{k}}\sin\left(\sqrt{\frac{k_2}{k}}l\right)&\cos\left(\sqrt{\frac{k_2}{k}}l\right)
        \end{bmatrix}\begin{bmatrix}
            r(0^+)\\
            r'(0^+)
        \end{bmatrix}=\begin{bmatrix}
            z_0r_0'\cos\left(\sqrt{\frac{k_2}{k}}l\right)+\sqrt{\frac{k}{k_2}}\frac{r_0'}{n_0}\sin\left(\sqrt{\frac{k_2}{k}}l\right)\\
            -\sqrt{\frac{k_2}{k}}z_0r_0'\sin\left(\sqrt{\frac{k_2}{k}}l\right)+\frac{r_0'}{n_0}\cos\left(\sqrt{\frac{k_2}{k}}l\right)
        \end{bmatrix}.
    \end{align}
    经出射面折射后, 光线可表为
    \begin{align}
        \begin{bmatrix}
            r(l^+)\\
            r'(l^+)
        \end{bmatrix}=\begin{bmatrix}
            1&0\\
            0&n_0
        \end{bmatrix}\begin{bmatrix}
            r(l^-)\\
            r'(l^-)
        \end{bmatrix}=\begin{bmatrix}
            z_0r_0'\cos\left(\sqrt{\frac{k_2}{k}}l\right)+\sqrt{\frac{k}{k_2}}\frac{r_0'}{n_0}\sin\left(\sqrt{\frac{k_2}{k}}l\right)\\
            -\sqrt{\frac{k_2}{k}}n_0z_0r_0'\sin\left(\sqrt{\frac{k_2}{k}}l\right)+r_0'\cos\left(\sqrt{\frac{k_2}{k}}l\right)
        \end{bmatrix}.
    \end{align}
    在 $z=l+d$ 处,
    \begin{align}
        r(l+d)=r(l^+)+dr'(l^+)=z_0r_0'\cos\left(\sqrt{\frac{k_2}{k}}l\right)+\sqrt{\frac{k}{k_2}}\frac{r_0'}{n_0}\sin\left(\sqrt{\frac{k_2}{k}}l\right)+\left[-\sqrt{\frac{k_2}{k}}n_0z_0r_0'\sin\left(\sqrt{\frac{k_2}{k}}l\right)+r_0'\cos\left(\sqrt{\frac{k_2}{k}}l\right)\right]d=0,
    \end{align}
    其中
    \begin{align}
        d=\frac{n_0z_0\cos\left(\sqrt{\frac{k_2}{k}}l\right)+\sqrt{\frac{k}{k_2}}\sin\left(\sqrt{\frac{k_2}{k}}l\right)}{\sqrt{\frac{k_2}{k}}n_0^2z_0\sin\left(\sqrt{\frac{k_2}{k}}l\right)-n_0\cos\left(\sqrt{\frac{k_2}{k}}l\right)},
    \end{align}
    注意到 $r(l+d)=0$ 不依赖于 $r_0'$, 故类透镜介质把轴上物点成像为单点.
\end{pf}
\ifx\allfiles\undefined
\end{document}
\fi
