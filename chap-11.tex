% !Tex program = pdflatex
% 第 11 章: 激光器的 Q 开关和锁模
\ifx\allfiles\undefined
\documentclass[twoside]{note}
\begin{document}
\fi
\setcounter{chapter}{10}
\chapter{激光器的 Q 开关和锁模}
\begin{exe}
    \begin{itemize}
        \item[(a)] 在锁模实验装置示意图 11.11 中的 A, B, C, D 部分预期可以观察到什么, 试定性地描述. 读者首先要阅读第 8 章参考文献 [7] 关于光电倍增管一节就可以找到答案.
        \item[(b)] 锁模对射频频谱分析仪 $F$ 所显示的拍频信号 (在频率为 $\omega=\pi c/l$ 处) 强度有什么影响? 假设 $N$ 是间隔为 $\omega$ 的等振幅模其位相在锁模前是随机的. (答: 锁模可使拍频信号功率增加 $N$ 倍.)
    \end{itemize}
\end{exe}
\begin{sol}
    \begin{itemize}
        \item[(a)] A 部分用可调光电倍增管直接探测锁模激光的信号, 可调光电倍增管对光信号具有显著的放大作用且响应速度很快, 故可得到时域上周期为 $\tau=\frac{T}{N}$ 的脉冲序列, 其中 $T=\frac{2l}{c}$, $N$ 为锁模的模式数.

        B 部分用光电倍增管探测锁模激光的信号, 并用射频频谱分析仪分析, 故可得一中心频率为 $\omega=\frac{2\pi}{T}=\frac{\pi c}{l}$ 的峰.

        C 部分先用扫描干涉仪选频, 再用光电倍增管探测透射频率的光的信号, 扫描干涉仪的本质是一 Fabry–Pérot 干涉仪, 通过扫描其腔长以改变其透射频率, 故可得锁模激光中的光谱, 即频域上介质增益曲线峰值点附近一系列间隔为 $\omega=\frac{2\pi}{T}=\frac{\pi c}{l}$ 的峰.

        D 部分用点接触二极管直接探测锁模激光的信号, 二极管对光信号的放大作用远不如光电倍增管, 故可得到时域上周期为 $T=\frac{2l}{c}$ 的脉冲序列.
        \item[(b)] 锁模前信号中存在 $\omega_0+n\omega$ ($n=0,\pm 1,\pm 2,\cdots$) 的各个频率分量, 锁模使各频率分量产生了干涉, 从而在时域上出现了明显的周期为 $T=\frac{2l}{c}$ 的拍频, 故锁模可使拍频信号功率增加 $N$ 倍.
    \end{itemize}
\end{sol}

\begin{exe}
    当介电常数 $\varepsilon$ (而不是损耗 $\sigma$) 在模之间的间隔频率 $c/2l$ 处受到调制时试分析锁模的情况. 可论述非均匀加宽激光器的情况, 在某种意义上类似于 11.2 节的形式, 或者论述 11.3 节所考虑的均匀加宽激光器的情况.
\end{exe}
\begin{sol}
    谐振腔内电磁场满足麦克斯韦方程:
    \begin{align}
        \nabla\times\bm{H}=&\sigma \bm{E}+\varepsilon\frac{\partial\bm{E}}{\partial t},\\
        \nabla\times\bm{E}=&-\mu\frac{\partial\bm{H}}{\partial t},
    \end{align}
    其中谐振腔内电磁场 $\bm{E}(\bm{r},t)$ 和 $\bm{H}(\bm{r},t)$ 可展开为简正模的线性叠加
    \begin{align}
        \bm{E}(\bm{r},t)=&-\sum_a\frac{1}{\sqrt{\varepsilon}}p_a(t)\bm{E}_a(\bm{r}),\\
        \bm{H}(\bm{r},t)=&\sum_a\frac{1}{\sqrt{\mu}}\omega_aq_a(t)\bm{H}_a(\bm{r}),
    \end{align}
    其中 $\omega_a=\frac{k_a}{\sqrt{\mu\varepsilon}}$,
    再利用关系式
    \begin{align}
        k_a\bm{E}_a=\nabla\times\bm{H}_a,\\
        k_a\bm{H}_a=\nabla\times\bm{E}_a,
    \end{align}
    得
    \begin{align}
        \label{11.2-1}
        \sum_a\frac{1}{\sqrt{\mu}}\omega_aq_ak_a\bm{E}_a=&-\frac{\sigma}{\sqrt{\varepsilon(\bm{r},t)}}\sum_ap_a\bm{E}_a-\sqrt{\varepsilon(\bm{r},t)}\sum_a\dot{p}_a\bm{E}_a,\\
        \label{11.2-2}
        \dot{q}_b=&p_b.
    \end{align}
    用 $\bm{E}_b$ 点乘式 \eqref{11.2-1} 并在腔体积范围内积分得
    \begin{align}
        \label{11.2-3}
        \omega_b^2q_b=-\sum_aS_{b,a}(t)p_a-\dot{p}_b,
    \end{align}
    其中
    \begin{align}
        \label{11.2-4}
        S_{b,a}(t)=\sigma\int_{\text{腔}}\frac{1}{\varepsilon(\bm{r},t)}\bm{E}_a\cdot\bm{E}_b\,\mathrm{d}v.
    \end{align}
    引入简正模振幅
    \begin{align}
        \label{11.2-normal-mode-amplitude}
        c_a(t)=(2\omega_a)^{-1/2}[\omega_aq_a(t)+ip_a(t)].
    \end{align}
    利用式 \eqref{11.2-normal-mode-amplitude} 及其在式 \eqref{11.2-2} 和 \eqref{11.2-3} 的复共轭
    \begin{align}
        \label{11.2-7}
        \frac{\mathrm{d}c_a}{\mathrm{d}t}=&-i\omega_ac_a+\sum_b\varkappa_{a,b}(t)(c_b^*-c_b),\\
        \label{11.2-8}
        \frac{\mathrm{d}c_a^*}{\mathrm{d}t}=&i\omega_ac_a^*-\sum_b\varkappa_{a,b}(t)(c_b^*-c_b),
    \end{align}
    其中
    \begin{align}
        \label{11.2-5}
        \varkappa_{a,b}(t)=\frac{1}{2}\sqrt{\frac{\omega_b}{\omega_a}}S_{a,b}(t).
    \end{align}
    取介电常数为一平均项和一谐波微扰项之和
    \begin{align}
        \varepsilon(\bm{r},0)=\varepsilon_0+\varepsilon_1(\bm{r})\cos(\omega_mt+\phi),
    \end{align}
    则利用式 \refeq{11.2-4} 和式 \eqref{11.2-5} 知 $\varkappa_{a,b}(t)$ 具有如下形式:
    \begin{align}
        \label{11.2-6}
        \varkappa_{a,b}(t)=\frac{\sigma_0}{2\varepsilon_0}\delta_{a,b}+\frac{\varkappa_{a,b}}{2}[e^{i(\omega_mt+\phi)}+e^{-i(\omega_mt+\phi)}],
    \end{align}
    其中
    \begin{align}
        \varkappa_{a,b}=-\frac{\sigma}{2\varepsilon_0^2}\sqrt{\frac{\omega_b}{\omega_a}}\int_{\text{腔}}\varepsilon_1(\bm{r})\bm{E}_a\cdot\bm{E}_b\,\mathrm{d}v.
    \end{align}
    将式 \eqref{11.2-6} 回代入运动方程 \eqref{11.2-7} 得
    \begin{align}
        \label{11.2-9}
        \frac{\mathrm{d}c_a^*}{\mathrm{d}t}=i\omega_ac_a^*+\frac{\sigma_0}{2\varepsilon_0}(c_a^*-c_a)-\sum_b\frac{\varkappa_{a,b}}{2}[e^{i(\omega_mt+\phi)}+e^{-i(\omega_mt+\phi)}](c_b^*-c_b).
    \end{align}
    定义失调参数 $\Delta\omega_m$ 为调试频率与模间距 (自由光谱范围) 的偏离值
    \begin{align}
        \omega_{a+1}-\omega_a=\frac{\pi c}{l}=\omega_m-\Delta\omega_m,
    \end{align}
    定义绝热变量 $D_a^*(t)$ 为
    \begin{align}
        c_a^*=D_a^*(t)e^{i[(\omega_a+a\Delta\omega_m)t+a\phi+a\pi/2]}e^{-(\sigma/2\varepsilon)t},
    \end{align}
    代入式 \eqref{11.2-9} 并略去相对于 $D_a^*(t)$ 的快变项得
    \begin{align}
        \frac{\mathrm{d}D_a^*}{\mathrm{d}t}+ia\Delta\omega D_a^*=-i\frac{\varkappa}{2}D_{a+1}^*+i\frac{\varkappa}{2}D_{a-1}^*,
    \end{align}
    其中 $\varkappa=\varkappa_{a,a+1}\approx\varkappa_{a,a-1}$.
    稳态下 ($\frac{\mathrm{d}D_a^*}{\mathrm{d}t}=0$), 解得绝热变量为
    \begin{align}
        D_a^*=I_a\left(\frac{\varkappa}{\Delta\omega}\right),
    \end{align}
    其中 $I_a$ 为 $a$ 阶双曲 Bessel 函数, 从而稳态下
    \begin{align}
        c_a^*(t)=I\left(\frac{\varkappa}{\Delta\omega}\right)e^{i[(\omega_a+a\Delta\omega_m)t+a\phi+a\pi/2]}e^{-\sigma t/2\varepsilon_0},
    \end{align}
    其中 $\omega_a+a\Delta\omega_m=\omega_0+a\frac{\pi c}{l}+a\Delta\omega_m=\omega_0+a\omega_m$, 当 $\frac{\varkappa}{\Delta\omega_m}\gg 1$, $I_a\left(\frac{\varkappa}{\Delta\omega_m}\right)$, 从而
    \begin{align}
        c_a^*(t)=\left(2\pi\frac{\varkappa}{\Delta\omega}\right)^{-1/2}e^{i[(\omega_0+a\omega_m)t+a\phi+a\pi/2]},
    \end{align}
    其中考虑到增益与损耗平衡, 故衰减因子 $\exp(-\sigma t/2\varepsilon_0)$.
    由上式可见, 各相邻振荡模式之间存在稳定的相位差 $\phi+\pi/2$, 从而可实现与调制 $\sigma$ 类似的锁模效果.
\end{sol}
\ifx\allfiles\undefined
\end{document}
\fi