% !Tex program = pdflatex
% 第 11 章: 激光器的 Q 开关和锁模
\ifx\allfiles\undefined
\documentclass{note}
\begin{document}
\setcounter{chapter}{10}
\fi
\chapter{激光器的 Q 开关和锁模}
\begin{exe}
    \begin{itemize}
        \item[(a)] 在锁模实验装置示意图 11.11 中的 $A$, $B$, $C$, $D$ 部分预期可以观察到什么, 试定性地描述. 读者首先要阅读第 8 章参考文献 [7] 关于光电倍增管一节就可以找到答案.
        \item[(b)] 锁模对射频频谱分析仪 $F$ 所显示的拍频信号 (在频率为 $\omega=\pi c/l$ 处) 强度有什么影响? 假设 $N$ 是间隔为 $\omega$ 的等振幅模其位相在锁模前是随机的. (答: 锁模可使拍频信号功率增加 $N$ 倍.)
    \end{itemize}
\end{exe}
\begin{sol}

\end{sol}

\begin{exe}
    当介电常数 $e$ (而不是损耗 $\sigma$) 在模之间的间隔频率 $c/2l$ 处受到调制时试分析锁模的情况. 可论述非均匀加宽激光器的情况, 在某种意义上类似于 11.2 节的形式, 或者论述 11.3 节所考虑的均匀加宽激光器的情况.
\end{exe}
\begin{sol}

\end{sol}
\ifx\allfiles\undefined
\end{document}
\fi